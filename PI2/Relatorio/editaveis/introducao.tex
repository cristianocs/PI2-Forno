\chapter*[Introdução]{Introdução}
\addcontentsline{toc}{chapter}{Introdução}

O tratamento térmico é o conjunto de operações de aquecimento e resfriamento a que são submetidos materiais sob condições controladas, com o objetivo de alterar suas propriedades ou conferir-lhes características mecânicas e estruturais diferentes.

Os tratamentos mais usuais são o recozimento, a normalização, a têmpera, o revenido e o coalescimento. Esses tratamentos são influenciados por alguns fatores que são: o aquecimento, tempo de permanência em determinada temperatura, resfriamento e atmosfera de aquecimento, portanto cada tratamento específico necessita do controle desses fatores.

Portanto um forno de tratamento térmico é de fundamental importância para se realizar esses tratamentos, logo para que o tratamento seja satisfatório, o forno deve ser capaz de variar a temperatura e manter a temperatura de forma satisfatória, de acordo com o tratamento requerido, sem que a temperatura do ambiente externo seja elevado a condições insalubres.


