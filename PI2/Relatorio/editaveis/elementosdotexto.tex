\chapter[Solução Proposta]{Solução Proposta}

\section{Justificativa}

\subsection{Especificação do Corpo de Prova}

\subsection{Tipo de Material}


\section{Dimensionamento do Forno}

O forno de tratamento térmico possuirá um volume interno de aproximadamente 8 litros, pois o objetivo é fabricar um forno de tratamento para fins acadêmicos, com corpos de prova segundo a norma ABNT 10611 e componentes mecânicos de pequeno porte, como engrenagens.
 
A parte externa será compostas por uma estrutura de aço carbono como um chassi de sustentação ao peso do forno bem como todo suporte para materiais elétricos/eletrônicos, esse envoltório também será usado em formato de placas de para cobrir os tijolos expostos ao ambiente externo a fim de proteger contra choque mecânicos, conforme demonstrado no anexo b

\begin{figure}[h]
	\centering
	\label{cad}
	\includegraphics[keepaspectratio=true,scale=0.8]{figuras/cad.JPG}
	\caption{Protótipo CAD do forno.}
\end{figure}

Os tijolos utilizados para tal construção possuem dimensões de 350 x 150 x 60 (mm) e são próprios para suportar temperaturas superiores a 1200 ºC conforme mostrado na tabela a seguir, concedida pela empresa fornecedora desse material.

\begin{figure}[h]
	\centering
	\label{tebela_dimensoes}
	\includegraphics[keepaspectratio=true,scale=1.0]{figuras/tabela_dimensoes.JPG}
	\caption{Características físicas e químicas do material.}
\end{figure}

\section{Material de Isolamento}

Notas eventualmente necessárias devem ser numeradas de forma seqüencial ao 
longo do texto no formato 1, 2, 3... sendo posicionadas no rodapé de cada 
página na qual a nota é utilizada.\footnote{Como, por exemplo, esta nota}


\section{Cálculo das Resistências}

As figuras devem ser centradas entre margens e identificadas por uma legenda 
alinhada a esquerda com recuo especial de deslocamento de 1,8 cm, com mostrado 
na Fig. (\ref{fig01}). O tamanho das fontes empregadas nos rótulos e anotações 
usadas nas figuras deve ser compatível com o usado no corpo do texto. Rótulos e 
anotações devem estar em português, com todas as grandezas mostradas em 
unidades do SI (Sistema Internacional de unidades).

Todas as figuras, gráficos e fotografias devem ser numeradas e referidas no 
corpo do texto adotando uma numeração seqüencial de identificação. As figuras e 
gráficos devem ser claras e com qualidade adequada para eventual reprodução 
posterior tanto em cores quanto em preto-e-branco.

As abscissas e ordenadas de todos os gráficos devem ser rotuladas com seus 
respectivos títulos em português seguida da unidade no SI que caracteriza a 
grandes entre colchetes. 

A referência explícita no texto à uma figura deve ser feita como 
\lq\lq Fig. (\ref{fig01})\rq\rq\ quando no meio de uma frase ou como 
\lq\lq Figura (\ref{fig01})\rq\rq\ quando no início da mesma. Referencias 
implícitas a uma dada figura devem ser feitas entre parênteses como 
(Fig. \ref{fig01}). Para referências a mais de uma figura as mesmas regras 
devem ser aplicadas usando-se o plural adequadamente. Exemplos:

\begin{itemize}
	\item \lq\lq Após os ensaios experimentais, foram obtidos os resultados 
	mostrados na Fig. (\ref{fig01}), que ...\rq\rq
	\item \lq\lq A Figura (\ref{fig01}) apresenta os resultados obtidos, onde 
	pode-se observar que ...\rq\rq
	\item \lq\lq As Figuras (1) a (3) apresentam os resultados obtidos, 
	...\rq\rq
	\item \lq\lq Verificou-se uma forte dependência entre as variáveis citadas 
	(Fig. \ref{fig01}), comprovando ...\rq\rq
\end{itemize}

Cada figura deve ser posicionada o mais próxima possível da primeira citação 
feita à mesma no texto, imediatamente após o parágrafo no qual é feita tal 
citação, se possível, na mesma página.
\begin{figure}[h]
	\centering
	\label{fig01}
		\includegraphics[keepaspectratio=true,scale=0.3]{figuras/fig01.jpg}
	\caption{Wavelets correlation coefficients}
\end{figure}

\section{Sistema de Alimentação}

\subsection{Tipo de Alimentação}

Para o funcionamento do forno de tratamento térmico foram estudadas duas formas de alimentação do sistema: 1) a gás e 2) via energia elétrica. Levando em consideração a disponibilidade e segurança, descartou-se a possibilidade da alimentação via gás, visto que o seu uso torna o sistema quanto ao manuseio periculoso, e além disso outro fator foi levado em consideração: restrição nos recursos financeiros orçados pelos alunos do projeto, e a indisponibilidade do recurso na Unidade Acadêmica de Ensino. Logo a escolha tomada para a geração de calor no forno de tratamento térmico se dará por meio da energia disponível e mais viável na execução do projeto: a energia elétrica. 

Como requisitos de segurança do sistema e da rede que conecta a tomada, o sistema será dimensionado para até 15 Amperes de corrente, pois de acordo com a NBR 14136, as tomadas domésticas devem operar a uma corrente de até 20 Amperes. Fazendo um fator de segurança de 5 amperes evitamos vários problemas com a sobre carga do circuito

\subsection{Sistema de Aquecimento}

Para que o interior do forno de tratamento térmico atinja altas temperaturas (até 1200 ), serão usados resistores para a geração de calor, os quais serão conectados a fonte de tensão a partir de um circuito de controle de corrente e estarão posicionados nas paredes internas do forno. Tais resistores se comportarão como fontes de calor para atmosfera interna do forno, e estarão dispostos de forma que tenha a maior área possível de contato para o ar e que exista um caminho livre para circulação e transferência do calor.

\subsection{Cálculo de Potência e Tempo de Aquecimento}

Para alcançar a potência desejada de 3kW dimensionou-se as propriedades termodinâmicas no volume de controle do forno.

\subsection{Dimensionamento e Resistência Elétrica}

\subsection{Distribuição dos Resistores}

Após saber o comprimento da resistência, a potência e a energia térmica necessária podemos dimensionar a distribuição dos resistores pelo volume interno do forno. Com os 30,73 metros calculados, será comprado uma quantidade de 35 metros de comprimento já enrolado que será distribuída pelas paredes do forno. 
A distribuição dependerá do tamanho final da peça a ser comprada, onde ela será acoplada as paredes a partir de canaletas furadas com um diâmetro um pouco menor que o da espiral da resistência para que ela fique bem fixa . O espaço entre as canaletas deve ser de uma distância segura entre elas como a distancia de um diâmetro da bobina. As resistências serão dispostas nas pareces laterais em formato de S em cada parede e terá seus terminais acoplados em três parafusos na parede de fundo do Forno. A ligação dos resistores com o circuito ocorrerá em paralelo, sendo os dois terminais ligados a dois parafusos na perfurados no fundo do forno que serão conectados ao circuito elétrico, e um terceiro parafuso que será enrolado com as outras extremidades das resistências transformando as duas em uma resistências só acoplada. Este parafuso não atravessará completamente o tijolo e não terá contato com a parte externa. Para segurança, uma caixa pequena de proteção será acoplada na parte traseira externa do forno para que não haja contato acidental com os parafusos e nem com o termostato.
Quando o operador tiver controle da quantidade de energia térmica do sistema ele será capaz de aquecer com eficiência um corpo de prova exposto a um tratamento térmico. Para testar a eficiência de controle térmico do forno e sua capacidade de executar um tratamento térmico será realizado uma têmpera do aço.
\begin{figure}[!h]
	\centering
	\label{parafusos}
	\includegraphics[keepaspectratio=true,scale=0.8]{figuras/parafusos.JPG}
	\caption{Esquema dos parafusos.}
\end{figure}

\subsection{Tratamento Térmico}

Um exemplo de tratamento térmico muito usado na industria é o processo de têmpera, que consiste no submeter o aço a uma temperatura em média 50 ºC acima da zona crítica de austenitização para o aços até 0,8\% de carbono e 50 ºC acima do limite inferior da zona crítica de austenitização para o aços acima de 0,8\% de carbono A zona crítica de austenitização varia com o aumento da composição de carbono no aço como na figura abaixo:
\begin{figure}[!h]
	\centering
	\label{austenitizacao}
	\includegraphics[keepaspectratio=true,scale=0.8]{figuras/austenitizacao.JPG}
	\caption{Zona de Austenitização para Têmpera. Fonte: Tschiptschin}
\end{figure}

Após manter o aço por um tempo  na zona crítica representada pela área vermelha do gráfico, ele muda sua sua estrutura cristalina para a austenítica ou “austenítica + cementita” que a altas temperaturas substituí qualquer estrutura existente no corpo de prova anteriormente.Após esse processo o aço é resfriado bruscamente em água passando de temperaturas entre 780 a 900 graus Celsius à temperatura ambiente de forma rápida com o objetivo de obter a estrutura Martensítica do aço que tem características duras e frágeis e evitando estruturas mais moles como ferrita, bainita e perlita:
\begin{figure}[!h]
	\centering
	\label{resfriamento1}
	\includegraphics[keepaspectratio=true,scale=0.8]{figuras/resfriamento1.JPG}
	\caption{Resfriamento de temperatura para o processo de têmpera. Fonte: Tschiptschin.}
\end{figure}

O tempo de formação de Martensita é diferente na superfície e no centro do corpo de prova, devido ao contato direto da superfície com o meio refrigerante, levando o tempo da temperização no centro do corpo de prova a ser até 10 vezes maior que na superfície. Essa variação temporal pode gerar diferenças na estrutura do material, e isso deve ser considerado dependendo do objetivo do operador.

\begin{figure}[!h]
	\centering
	\label{resfriamento2}
	\includegraphics[keepaspectratio=true,scale=0.8]{figuras/resfriamento2.JPG}
	\caption{Resfriamento na superfície no centro do corpo de prova. Fonte: Tschiptschin.}
\end{figure}

\subsection{Material Submetido ao Tratamento}

O aço é definido no Brasil pela NBR 6215:2011 como uma liga ferrosa passível de deformação plástica que, em geral, apresenta teor de carbono entre 0,008\% e 2,0\% na sua forma combinada e, ou, dissolvida e que pode conter elementos de liga adicionados, ou residuais.

No Brasil, a Associação Brasileira de Normas Técnicas - ABTN, por intermédio da norma NBR NM 87:2000 classifica os aços-carbono comuns e os de baixo teor em liga segundo os critérios adotados pela AISI (\textit{American Iron and Steel Institute}) e SAE (\textit{Society of Automotives Engineers}).

\begin{figure}[!h]
	\centering
	\label{tab_sae1}
	\includegraphics[keepaspectratio=true,scale=0.8]{figuras/tab_sae1.JPG}
	\caption{Composição química de aços Sae 10XX. Fonte: ABNT/SAE J403, 1995.}
\end{figure}

\begin{figure}[!h]
	\centering
	\label{tab_sae2}
	\includegraphics[keepaspectratio=true,scale=0.8]{figuras/tab_sae2.JPG}
	\caption{Características de tratamento térmico de aços-carbono simples.  Fonte: ABNT/SAE J403, 1995.}
\end{figure}

Aços são considerados bons para serem temperados a partir de 40\% de carbono em sua composição, pois possuem resistência suficiente ao desgaste a abrasão e boa tenacidade.

Para o trabalho foi escolhido o corpo de prova SAE 1045, que é considerado aço de médio teor de carbono, por questões de preço, disponibilidade, esfriamento em água facilitado e faixa de dureza próxima à aços com maior porcentagem de carbono.
São aços que possuem boa conformabilidade à frio e razoável resistência mecânica com acabamento laminado, trefilado ou retificado. 

Para que seja realizados tratamentos térmicos, é necessário ter conhecimento sobre a curva TTT (tempo-temperatura-transformação) do material, que relaciona as variáveis da micro estrutura com o tempo e a temperatura.
\begin{figure}[!h]
	\centering
	\label{diagramattt}
	\includegraphics[keepaspectratio=true,scale=0.8]{figuras/diagramattt.JPG}
	\caption{Diagrama TTT do aço SAE 1045 Fonte: DOMINGUES et.al 2010.}
\end{figure}

\begin{figure}[!h]
	\centering
	\label{curva_temperatura}
	\includegraphics[keepaspectratio=true,scale=0.8]{figuras/curva_temperatura.JPG}
	\caption{Curva de temperabibilidade do aço SAE 1045. Fonte: MARTINS, 2002.}
\end{figure}

O aço Sae 1045 costuma ter no máximo seção transversal de no máximo 60mm. Para seções transversais maiores, o material não apresenta boa reação à têmpera e sua dureza diminui sensivelmente. O aço SAE 1045 deve ser aquecido entre 820ºC e 840ºC em média, por 10 minutos por milímetro, para que haja maior elevação da ductilidade e resistência assim como evitar trincamentos.

O processo de têmpera é completo após a técnica de resfriamento do material, após a temperatura de austenitização, no qual objetiva-se a formação de constituintes resultantes como cementita, ferrita e principalmente martensita, fase metaestável supersaturada de carbono e, portanto de alta dureza. O rápido processo de troca de ambiente de alto calor para um ambiente de baixo calor (até um segundo) permite a formação de martensita.

\begin{figure}[!h]
	\centering
	\label{tab_valoresH}
	\includegraphics[keepaspectratio=true,scale=0.8]{figuras/tab_valoresH.JPG}
	\caption{Valores de H (coeficientes de severidade de têmpera) para diferentes meios de têmpera. FONTE: SCHEIDEMANTEL, 2014}
\end{figure}

A variação da taxa de resfriamento entre água e salmoura é de 27,6\% até 110\% e a diminuição do tempo de resfriamento é de 7,8\% até 63,3\% em relação à agitação. Ar e óleo possui uma severidade muito baixa de tempera e salmoura possui uam severidade mais alta que o ideal. Dessa forma o resultado de têmpera com resfriamento em água gera um aumento de 20\% de martensita na estrutura do aço em relação á tratamento com salmoura. (CARVALHO, 2004)


\section{Termodinânica}

Referencias a outros trabalhos tais como artigos, teses, relatórios, etc. devem 
ser feitas no corpo do texto devem estar de acordo com a norma corrente ABNT 
NBR 6023:2002 (ABNT, 2000), esta ultima baseada nas normas ISO 690:1987:
\begin{itemize}
	\item \lq\lq \cite{bordalo1989}, mostraram que...\rq\rq

	\item \lq\lq Resultados disponíveis em \cite{coimbra1978}, \cite{clark1986} 
	e \cite{sparrow1980}, mostram que...\rq\rq
\end{itemize}

Para referências a trabalhos com até dois autores, deve-se citar o nome de 
ambos os autores, por exemplo: \lq\lq \cite{soviero1997}, mostraram 
que...\rq\rq

\section{Termodinânica}

\section{Sistema de Controle}

O sistema de controle do forno será constituído por um conjunto de módulos de hardware e software em malha fechada, ou seja, as informações de saída influenciam no comportamento interno do sistema. O software será responsável pela interação do usuário com o forno, no qual o usuário irá determinar a temperatura de atuação desejada. Já o hardware será responsável por controlar as grandezas do sistema como corrente, tensão, temperatura.

\begin{figure}[h]
	\centering
	\label{diagrama}
	\includegraphics[keepaspectratio=true,scale=0.5]{figuras/diagrama.jpg}
	\caption{Sistema de controle de temperatura do forno}
\end{figure}

O módulo da Raspberry Pi será responsável por gerenciar um sistema de cadastro de usuários, onde por meio de seu login pessoal será possível a utilização do forno. Esse sistema registrará por meio de log de eventos, as ações realizadas no forno por este usuário específico.

Além disso, esse sistema será o responsável por dispor relatórios com os acontecimentos do experimento realizado como gráfico de tempo X temperatura. Estes relatórios serão salvos em um banco de dados, com o intuito do usuário poder consultar os seus experimentos e de outros usuários que possam ter sido feitos anteriormente, tendo assim um histórico de experimentos.

Devido as especificações técnicas dadas, o sensor termopar tipo K foi escolhido, já que possui uma grande amplitude de atuação (valores típicos entre -270 e 1230 ºC). O termopar gera uma diferença de potencial de acordo com a variação de temperatura devido as características de seu material. Com isso, para iniciar o controle da temperatura do forno, inicialmente deve-se fazer uma calibração do termopar. Essa etapa consiste em determinar os valores de tensão de saída do termopar para temperaturas específicas com o intuito de se obter a relação tensão X temperatura.

Para que a temperatura do termopar possa ser obtida de maneira adequada, um circuito de amplificação será utilizado para que o seu sinal, que tem uma variação padrão de aproximadamente 41$\mu$V/ºC, possa ter uma diferença significante para a resolução do conversor A/D. Esse circuito também é responsável por uma compensação de junção fria, necessária para retirar a temperatura ambiente do cálculo da temperatura.

Por fim, a tensão final é amostrada na entrada do conversor AD, que estará conectado à Raspberry Pi e irá transmitir os bits referentes a tensão lida através de um canal de comunicação SPI,sincronizado pelo clock definido pela Raspberry Pi. A Raspberry Pi irá filtrar os valores de temperatura por software. Ela também é responsável por setar a temperatura desejada. Isso é realizado através do envio dos bits através dos pinos de I/O por comunicação SPI à um conversor D/A, que irá converter para a tensão desejada que irá alimentar o circuito de controle. Esse valor de tensão representa a temperatura que deve ser lida pelos sensores com o tempo.

Ao fazer o login o usuário irá determinar qual temperatura o forno deverá ser submetido. A Raspberry Pi irá emitir um sinal digital, uma sequência de bits por exemplo, que deverá ser convertido em um sinal analógico, um valor de tensão conforme a curva definida pelo termopar. Para isso deve-se utilizar um conversor D/A, no qual sua saída será encaminhada ao módulo de controle do sistema.

Para manter o sistema na temperatura desejada será utilizado um controlador ON/OFF. De acordo com a tabela (x – CARACTERISTICAS DE TRATAMENTO TÉRMICO DO AÇO), tem-se que a maior variação da temperatura de austenitização é de 40ºC. Essa variação irá definir uma margem de erro, diferença entre o valor máximo e o valor mínimo, que a temperatura deverá se manter. Por exemplo, caso se defina o tratamento para o aço 1040, sua temperatura de austenitização está entre 790ºC e 870ºC, com a margem de erro de 40ºC, a temperatura que o usuário deverá escolher para garantir a têmpera está entre 810ºC e 850ºC. Na figura abaixo foi feito uma simulação de qual resposta se espera do módulo do controlador para uma temperatura de 300ºC.

\begin{figure}[h]
	\centering
	\label{on_off}
	\includegraphics[keepaspectratio=true,scale=0.8]{figuras/on_off.JPG}
	\caption{Simulação do sistema utilizando um controlador ON/OFF.}
\end{figure}

A principal desvantagem do controlador ON/OFF consiste na repetida variação da saída de controle de maneira rápida e bruta, o que pode acarretar no mau funcionamento do sistema. Porém, como o forno demora um tempo considerável para atingir uma temperatura, sabe-se que esse tempo é grande o suficiente para garantir que a repetição da saída de controle será lenta, fazendo com que a saída do controlador seja feita de forma suave.

A saída do controlador irá permitir ou bloquear (ligar ou desligar) a passagem de corrente para o sistema de aquecimento do forno através de um relé. Abaixo pode-se ver o sinal de controle gerado para o mesmo sistema mostrado na figura (\ref{saida_controle}).

\begin{figure}[h]
	\centering
	\label{saida_controle}
	\includegraphics[keepaspectratio=true,scale=0.8]{figuras/saida_controle.JPG}
	\caption{Saída de controle do controlador ON/OFF para o relé.}
\end{figure}

\section{Arquitetura de Software}

Para a construção da aplicação proposta foi definido a implementação de um sistema web ao qual ficarão dispostas todas as funcionalidades. Este sistema terá como servidor a Raspberry Pi e múltiplos usuários poderão se conectar e visualizar o processo de têmpera.

Com isso, a implementação deste sistema utilizará das seguintes tecnologias:
\begin{itemize}
	\item Linguagem de Programação \textbf{Python} - linguagem de programação de alto nível, multiparadigma, interpretada e de tipagem dinâmica e forte;
	\item \textbf{Django Framework} - framework para desenvolvimento web, escrito em Python, open source, que utiliza o padrão model-template-view. (MTV), utiliza por padrão banco de dados Sqlite3;
	\item \textbf{REST Framework} -  ferramenta open source poderosa e flexível para a construção de APIs Web.
	\item \textbf{ReactJS} - biblioteca open source de JavaScript fornecendo uma visão para os dados processados em HTML. Utiliza-se do conceito de componentes, aos quais permite uma modularização da página web, diminuindo assim sua interdependência e aumentando o reuso.
	
\end{itemize}

\subsection{Metodologia de Desenvolvimento}

A metodologia do projeto de software do forno deverá seguir as seguintes práticas:
\begin{itemize}
	\item Elicitação de requisitos: Serão utilizados métodos de brainstorming e prototipação rápida para aquisição de novas features. Cada feature nova será escrita em um cartão com uma linguagem simples para que qualquer pessoa, sendo desenvolvedora ou não, consiga entender do que se trata. Em cada cartão haverá também valor de prioridade e critérios de aceitação para que o seja validada.
	\item Cartões: os cartões seguem dois modelos. O primeiro, criado em 2001 por uma equipe de desenvolvedores da empresa Connextra. Deve possuir título auto explicativo e a descrição seguindo o formato “Como <papel>, gostaria que <desejo/meta>, de modo que <benefício>”. O segundo é o modelo dos três C’s que foi idealizado por Ron Jeffries também em 2001. Além de conter a descrição do cartão no formato acima, juntamente com o valor de prioridade, ainda há espaços para conversação entre desenvolvedores e clientes e, por fim, os critérios de aceitação que é a confirmação de que o que foi desenvolvido está de acordo com o descrito.
	\item Kanban: ferramenta para indicar o andamento do fluxo do processo de desenvolvimento do software, permite um controle detalhado de em qual etapa se encontra a funcionalidade que esteja sendo desenvolvida. Constitui de um quadro com 4 colunas em geral: Backlog - onde ficam as funcionalidades à serem desenvolvidas; A fazer (To do) - selecionadas as funcionalidades definidas para aquela determinada Sprint, elas são colocadas nesta coluna; Em desenvolvimento (Doing) - funcionalidade em desenvolvimento; Concluída (Done) - após a finalização do desenvolvimento da funcionalidade, esta é posta nesta coluna. Assim sendo, esta ferramenta permite uma visualização melhor do fluxo de trabalho no desenvolvimento da aplicação.
	\item Sprints: é um período de tempo variável entre uma semana a um mês onde as funcionalidades escolhidas são desenvolvidas. No início de cada sprint, uma sprint planning meeting é realizada informalmente para se fazer uma retrospectiva da sprint que passou e planejamento da próxima que está começando.
	
\end{itemize}

\subsection{\textit{Features}}

As features do sistema de controle do forno térmico serão descritas pela matriz Feature e Benefício (FAB) \cite{safe2015}. Elas descrevem os requisitos funcionais do sistema proposto.

\begin{table}[h]
	\centering
	\label{my-label}
	\begin{tabular}{|l|l|}
		\hline
		\multicolumn{1}{|c|}{\textbf{Features}} & \multicolumn{1}{c|}{\textbf{Benefícios}}                                                                                                                             \\ \hline
		Cadastro de Usuários                    & Possibilitar o cadastro de quem poderá utilizar o forno.                                                                                                             \\ \hline
		Sistema de Login                        & Garantir a segurança do sistema, identificando o usuário que utilizará o forno.                                                                                      \\ \hline
		Seletor de Temperatura                  & Possibilitar ao usuário escolher a temperatura desejada para a realização do experimento.                                                                            \\ \hline
		Cronômetro                              & \begin{tabular}[c]{@{}l@{}}Informa o tempo decorrido do experimento, o tempo esperado para término e o tempo restante para chegar no\\   esperado.\end{tabular}      \\ \hline
		Gráfico de Temperatura                  & Apresentar ao usuário em tempo real a variação de temperatura ocorrida dentro do forno.                                                                              \\ \hline
		Iniciar Processo de Têmpera             & Permitir ao usuário iniciar o processo ao que se dará o experimento.                                                                                                 \\ \hline
		Histórico de Sessão                     & Salvar relatório dos experimentos anteriormente realizados, e apresentá-los caso solicitados.                                                                        \\ \hline
		Sistema de Segurança                    & Dispor de um botão-emergência ao qual deverá interromper o processo imediatamente.                                                                                   \\ \hline
		Informações                             & \begin{tabular}[c]{@{}l@{}}Mostrar informações sobre o processo à de têmpera, assim como as informações necessárias para realização\\   do experimento.\end{tabular} \\ \hline
		NBRs                                    & Apresentar ao usuário as normas relacionadas ao procedimento e materiais.                                                                                            \\ \hline
		Estatísticas                            & Por meio dos dados de experimentos anteriores, apresentar estatísticas sobre a utilização do forno.                                                                  \\ \hline
	\end{tabular}
	\caption{Matrix de \textit{Features}}
\end{table}